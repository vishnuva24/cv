\documentclass[10pt, letterpaper]{article}

% Packages:
\usepackage[
    ignoreheadfoot, % set margins without considering header and footer
    top=2 cm, % seperation between body and page edge from the top
    bottom=1 cm, % seperation between body and page edge from the bottom
    left=2 cm, % seperation between body and page edge from the left
    right=2 cm, % seperation between body and page edge from the right
    footskip=1.0 cm, % seperation between body and footer
    % showframe % for debugging 
]{geometry} % for adjusting page geometry
\usepackage{titlesec} % for customizing section titles
\usepackage{tabularx} % for making tables with fixed width columns
\usepackage{array} % tabularx requires this
\usepackage[dvipsnames]{xcolor} % for coloring text
\definecolor{primaryColor}{RGB}{0, 79, 144} % define primary color
\usepackage{enumitem} % for customizing lists
\usepackage{fontawesome5} % for using icons
\usepackage{amsmath} % for math
\usepackage[
    pdftitle={Sujatha Varadarajan's CV},
    pdfauthor={Sujatha Varadarajan},
    pdfcreator={LaTeX with RenderCV},
    colorlinks=true,
    urlcolor=primaryColor
]{hyperref} % for links, metadata and bookmarks
\usepackage[pscoord]{eso-pic} % for floating text on the page
\usepackage{calc} % for calculating lengths
\usepackage{bookmark} % for bookmarks
\usepackage{datetime}
\usepackage{lastpage} % for getting the total number of pages
\usepackage{changepage} % for one column entries (adjustwidth environment)
\usepackage{paracol} % for two and three column entries
\usepackage{ifthen} % for conditional statements
\usepackage{needspace} % for avoiding page brake right after the section title
\usepackage{iftex} % check if engine is pdflatex, xetex or luatex

% Ensure that generate pdf is machine readable/ATS parsable:
\ifPDFTeX
    \input{glyphtounicode}
    \pdfgentounicode=1
    % \usepackage[T1]{fontenc} % this breaks sb2nov
    \usepackage[utf8]{inputenc}
    \usepackage{lmodern}
\fi



% Some settings:
\AtBeginEnvironment{adjustwidth}{\partopsep0pt} % remove space before adjustwidth environment
\pagestyle{empty} % no header or footer
\setcounter{secnumdepth}{0} % no section numbering
\setlength{\parindent}{0pt} % no indentation
\setlength{\topskip}{0pt} % no top skip
\setlength{\columnsep}{0cm} % set column seperation
\makeatletter
\let\ps@customFooterStyle\ps@plain % Copy the plain style to customFooterStyle
% \patchcmd{\ps@customFooterStyle}{\thepage}{
%     \color{gray}\textit{\small Vishnu Varadarajan - Page \thepage{} of \pageref*{LastPage}}
% }{}{} % replace number by desired string
% \makeatother
\patchcmd{\ps@customFooterStyle}{\thepage}{}{}{} % replace number by desired string
\pagestyle{customFooterStyle}

\titleformat{\section}{\needspace{4\baselineskip}\bfseries\large}{}{0pt}{}[\vspace{1pt}\titlerule]

\titlespacing{\section}{
    % left space:
    -1pt
}{
    % top space:
    0.3 cm
}{
    % bottom space:
    0.2 cm
} % section title spacing

\renewcommand\labelitemi{$\circ$} % custom bullet points
\newenvironment{highlights}{
    \begin{itemize}[
        topsep=0.10 cm,
        parsep=0.10 cm,
        partopsep=0pt,
        itemsep=0pt,
        leftmargin=0.4 cm + 10pt
    ]
}{
    \end{itemize}
} % new environment for highlights

\newenvironment{highlightsforbulletentries}{
    \begin{itemize}[
        topsep=0.10 cm,
        parsep=0.10 cm,
        partopsep=0pt,
        itemsep=0pt,
        leftmargin=10pt
    ]
}{
    \end{itemize}
} % new environment for highlights for bullet entries


\newenvironment{onecolentry}{
    \begin{adjustwidth}{
        0.2 cm + 0.00001 cm
    }{
        0.2 cm + 0.00001 cm
    }
}{
    \end{adjustwidth}
} % new environment for one column entries

\newenvironment{twocolentry}[2][]{
    \onecolentry
    \def\secondColumn{#2}
    \setcolumnwidth{\fill, 4.5 cm}
    \begin{paracol}{2}
}{
    \switchcolumn \raggedleft \secondColumn
    \end{paracol}
    \endonecolentry
} % new environment for two column entries

\newenvironment{header}{
    \setlength{\topsep}{0pt}\par\kern\topsep\centering\linespread{1.5}
}{
    \par\kern\topsep
} % new environment for the header

\newcommand{\placelastupdatedtext}{% \placetextbox{<horizontal pos>}{<vertical pos>}{<stuff>}
  \AddToShipoutPictureFG*{% Add <stuff> to current page foreground
    \put(
        \LenToUnit{\paperwidth-2 cm-0.2 cm+0.05cm},
        \LenToUnit{\paperheight-1.0 cm}
    ){\vtop{{\null}\makebox[0pt][c]{
        \small\color{gray}\textit{Last updated in \monthname \ \the\year}\hspace{\widthof{Last updated in June 2025}}
    }}}%
  }%
}%

% save the original href command in a new command:
\let\hrefWithoutArrow\href

% new command for external links:
\renewcommand{\href}[2]{\hrefWithoutArrow{#1}{\ifthenelse{\equal{#2}{}}{ }{#2 }\raisebox{.15ex}{\footnotesize \faExternalLink*}}}


\begin{document}
    \newcommand{\AND}{\unskip
        \cleaders\copy\ANDbox\hskip\wd\ANDbox
        \ignorespaces
    }
    \newsavebox\ANDbox
    \sbox\ANDbox{}

    \placelastupdatedtext
    \begin{header}
        \textbf{\fontsize{24 pt}{24 pt}\selectfont Sujatha Varadarajan}

        \vspace{0.3 cm}

        \normalsize
        \mbox{{\color{black}\footnotesize\faMapMarker*}\hspace*{0.13cm}Pune}%
        \kern 0.25 cm%
        \AND%
        \kern 0.25 cm%
        \mbox{\hrefWithoutArrow{mailto:sujsvarada@gmail.com}{\color{black}{\footnotesize\faEnvelope[regular]}\hspace*{0.13cm}sujsvarada@gmail.com}}%
        \kern 0.25 cm%
        \AND%
        \kern 0.25 cm%
        \mbox{\hrefWithoutArrow{tel:+91 8888806426}{\color{black}{\footnotesize\faPhone*}\hspace*{0.13cm}+91 8888806426}}%
        \kern 0.25 cm%
        \AND%
        \kern 0.25 cm%
        % \mbox{\hrefWithoutArrow{https://yourwebsite.com/}{\color{black}{\footnotesize\faLink}\hspace*{0.13cm}yourwebsite.com}}%
        % \kern 0.25 cm%
        % \AND%
        % \kern 0.25 cm%
        % \mbox{\hrefWithoutArrow{https://linkedin.com/in/yourusername}{\color{black}{\footnotesize\faLinkedinIn}\hspace*{0.13cm}yourusername}}%
        % \kern 0.25 cm%
        % \AND%
        % \kern 0.25 cm%
        \mbox{\hrefWithoutArrow{https://in.linkedin.com/in/dr-sujatha-varadarajan-7006216}{\color{black}{\footnotesize\faLinkedin}\hspace*{0.13cm}Linkdin}}%
    \end{header}

    \vspace{0.3 cm - 0.1 cm}


    A versatile educationist with a strong passion for teaching and enhancing the teaching-learning process. Brings extensive experience in pedagogical design, curriculum development, and organizing outreach programs and faculty development initiatives. Skilled in research design, undergraduate course development, and multidisciplinary curriculum planning, with a keen interest in educational innovation and academic community building.

    
    \section{Education}

        \begin{twocolentry}{
            \textit{Ph.D., Feb 2024}}
            \textbf{Tata Institute of Fundamental Research (TIFR), Mumbai}

            \textit{Ph.D. in Science Education. Awarded Best Thesis Commendation.}
        \end{twocolentry}

        \vspace{0.2cm}
        \begin{onecolentry}
            \begin{highlights}
                \item \textbf{Thesis:} Problem-based Learning in Undergraduate Chemistry Laboratories in India
            \end{highlights}
        \end{onecolentry}

        \vspace{0.20 cm}

        \begin{twocolentry}{
            \textit{Jun 1991 - Mar 1993}}
            \textbf{Madras University}

            \textit{M.Sc. in Chemistry (University First Rank Holder)}
        \end{twocolentry}

        \vspace{0.20 cm}

        \begin{twocolentry}{
            \textit{}}
            \textbf{Bharathiar University}

            \textit{M.Sc. in Applied Psychology}
        \end{twocolentry}





    
    \section{Experience}
    \begin{twocolentry}{
        \textit{}    
            
        \textit{Jan 2025 - May 2025}}
            \textbf{Centre Head, Multidisciplinary Curriculum and Pedagogy}
            
            \textit{Maharashtra State Faculty Development Academy, (MSFDA)}
        \end{twocolentry}

        \vspace{0.10 cm}
        \begin{onecolentry}
            \begin{highlights}
                \item As Centre Head, I spearhead faculty development programs for teachers all across Maharashtra through formal collaboration with many institutes of eminence and national importance. 
                \item The position also demands academic inputs for the design of in-house programs in general, and multidisciplinary curriculum design in particular. 
                \item Development of online content is one of the most important tasks undertaken as centre head currently.
            \end{highlights}
        \end{onecolentry}


        \vspace{0.2 cm}


        \begin{twocolentry}{
\textit{Varanasi, India}    

\textit{1993 – 1996}}
    \textbf{Research Scholar}
    
    \textit{Indian Institute of Technology (IT-BHU)}
\end{twocolentry}

\vspace{0.10 cm}
\begin{onecolentry}
    \begin{highlights}
        \item Conducted research in applied chemistry and toxicology.
        \item Brief research stints at Central Leather Research Institute, Chennai and Indian Toxicological Research Centre, Lucknow.
        \item Published three papers, including two in conference proceedings.
    \end{highlights}
\end{onecolentry}

\vspace{0.2 cm}

% --------- Graduate Student at HBCSE ---------
\begin{twocolentry}{
\textit{Mumbai, India}    

\textit{2018 – 2023}}
    \textbf{Graduate Student – Science Education}
    
    \textit{Homi Bhabha Centre for Science Education (TIFR)}
\end{twocolentry}

\vspace{0.10 cm}
\begin{onecolentry}
    \begin{highlights}
        \item Conducted doctoral research on problem-based learning in undergraduate chemistry labs across India.
        \item Designed pedagogical models emphasizing scaffolding and reflective thinking.
    \end{highlights}
\end{onecolentry}

\vspace{0.2 cm}

% --------- Visiting Faculty at SPPU ---------
\begin{twocolentry}{
\textit{Pune, India}    

\textit{2019 – 2022}}
    \textbf{Visiting Faculty – Chemistry}
    
    \textit{Savitribai Phule Pune University}
\end{twocolentry}

\vspace{0.10 cm}
\begin{onecolentry}
    \begin{highlights}
        \item Developed and implemented an inquiry-based chemistry laboratory course.
        \item Taught Environmental Toxicology to undergraduate and graduate students.
    \end{highlights}
\end{onecolentry}

\vspace{0.2 cm}

% --------- Teaching Assistant at IISER ---------
\begin{twocolentry}{
\textit{Pune, India}    

\textit{2012 – 2014}}
    \textbf{Teaching Assistant}
    
    \textit{Indian Institute of Science Education and Research (IISER)}
\end{twocolentry}

\vspace{0.10 cm}
\begin{onecolentry}
    \begin{highlights}
        \item Supported teaching in trans-disciplinary courses such as Rational Inquiry.
        \item Designed and led inquiry-based science workshops for high school students.
        \item Facilitated science education webinars for students and educators.
    \end{highlights}
\end{onecolentry}

\vspace{0.2 cm}

% --------- Consultant at MVS Science Center ---------
\begin{twocolentry}{
\textit{Pune, India}    

\textit{2007 – 2012}}
    \textbf{Consultant – Science Communication}
    
    \textit{MVS Science Center, IUCAA}
\end{twocolentry}

\vspace{0.10 cm}
\begin{onecolentry}
    \begin{highlights}
        \item Designed science communication activities for schools and public engagement.
        \item Helped establish hands-on science activities in various high schools.
        \item Conducted workshops for teachers, students, and the public.
    \end{highlights}
\end{onecolentry}

\vspace{0.2 cm}

% --------- Blue Bells School ---------
\begin{twocolentry}{
\textit{Gurgaon, India}    

\textit{2002 – 2005}}
    \textbf{TGT/PGT Science and Chemistry Teacher}
    
    \textit{Blue Bells Model Public School}
\end{twocolentry}

\vspace{0.10 cm}
\begin{onecolentry}
    \begin{highlights}
        \item Taught Science and Math at middle school level.
        \item Taught Chemistry at high school and higher secondary levels.
        \item Guided students in inter-school science activities and competitions.
    \end{highlights}
\end{onecolentry}

\vspace{0.2 cm}

% --------- St. Joseph’s School ---------
\begin{twocolentry}{
\textit{Bhopal, India}    

\textit{1999 – 2002}}
    \textbf{Primary Teacher – Science and Math}
    
    \textit{St. Joseph’s Senior Secondary School}
\end{twocolentry}

\vspace{0.10 cm}
\begin{onecolentry}
    \begin{highlights}
        \item Taught Science and Math to primary-level students.
        \item Integrated inquiry-based approaches in early science education.
    \end{highlights}
\end{onecolentry}

\vspace{0.2 cm}

\vspace{0.2 cm}

% --------- Workshops and Invited Talks ---------
\begin{twocolentry}{
\textit{Various Institutions}    

\textit{2007 – Present}}
    \textbf{Workshops and Invited Talks on Inquiry-Based Science Teaching}
    
    \textit{Colleges, Schools, and Conferences}
\end{twocolentry}

\vspace{0.10 cm}
\begin{onecolentry}
    \begin{highlights}
        \item Conducted workshops on inquiry-based science teaching at Garware College, Pune; IISER Pune; Modern College, Pune; Vaze College, Mumbai; St. Felix School, Pune; Vidya Niketan, Pune; Scinnovity, Pune, 
        \item delivered invited talks on the teaching-learning process at Pratibha College, Pune; National Safety Organization, Pune; ChemCollective Virtual Workshop, IIT Bombay.   
    \end{highlights}
\end{onecolentry}

\vspace{0.2 cm}

\begin{twocolentry}{
\textit{India and International}    

\textit{Ongoing}}
    \textbf{Editorial and Reviewing Roles, Faculty Facilitation}
    
    \textit{}
\end{twocolentry}

\vspace{0.10 cm}
\begin{onecolentry}
    \begin{highlights}
        \item Reviewer for \textit{Journal of Chemical Education} (Impact Factor: 2.5)
        \item Reviewer for \textit{Chemistry Education Research and Practice} (Impact Factor: 2.6)
        \item Editor for \textit{Science Shore}, an e-magazine focused on science education
        \item Facilitator in Faculty Development Programs across various colleges in Maharashtra
    \end{highlights}
\end{onecolentry}


    \section{Publications}

\begin{samepage}
    \begin{twocolentry}{Aug 2024}
        \textbf{Problem-Based Learning (PBL): A literature review of theory and practice in undergraduate chemistry laboratories}
            \vspace{0.10 cm}


        \mbox{\textbf{\textit{S. Varadarajan}}, S. Ladage}
    \end{twocolentry}

    \begin{onecolentry}
        \href{https://doi.org/10.1021/acs.jchemed.3c01335}{10.1021/acs.jchemed.3c01335}
    \end{onecolentry}
\end{samepage}

\vspace{0.20 cm}

\begin{samepage}
    \begin{twocolentry}{Jun 2023}
        \textbf{Doing Science: Introducing Scientific Inquiry to Elementary Classes}

            \vspace{0.10 cm}


        \mbox{\textbf{\textit{S. Varadarajan}}, S. Sanzgiri}
    \end{twocolentry}

    \begin{onecolentry}
        \textit{Resonance}, 28(6), 929–944
    \end{onecolentry}
\end{samepage}
    \vspace{0.20 cm}

\begin{samepage}
    \begin{twocolentry}{Dec 2022}
        \textbf{Introducing Incremental Levels of Inquiry in an Undergraduate Chemistry Laboratory: A Case Study on a Short Lab Course}

            \vspace{0.10 cm}

        \mbox{\textbf{\textit{S. Varadarajan}}, S. Ladage}
    \end{twocolentry}

    \begin{onecolentry}
        \textit{Journal of Chemical Education}, 99(12), 3822–3832
    \end{onecolentry}
\end{samepage}
    \vspace{0.20 cm}

\begin{samepage}
    \begin{twocolentry}{2022}
        \textbf{Problem-Based Learning in Undergraduate Chemistry Laboratories in India}

            \vspace{0.10 cm}

        \mbox{\textbf{\textit{S. Varadarajan}}}
    \end{twocolentry}

    \begin{onecolentry}
        Doctoral Dissertation, Tata Institute of Fundamental Research, Mumbai
    \end{onecolentry}
\end{samepage}
    \vspace{0.20 cm}

\begin{samepage}
    \begin{twocolentry}{2021}
        \textbf{Exploring the Role of Scaffolds in Problem-Based Learning (PBL) in an Undergraduate Laboratory}

            \vspace{0.10 cm}

        \mbox{\textbf{\textit{S. Varadarajan}}, S. Ladage}
    \end{twocolentry}

    \begin{onecolentry}
        \href{https://doi.org/10.1039/D1RP00180A}{10.1039/D1RP00180A}
    \end{onecolentry}
\end{samepage}
    \vspace{0.20 cm}

\begin{samepage}
    \begin{twocolentry}{2021}
        \textbf{Adapting an Inquiry-Based Approach for Undergraduate Chemistry Laboratory: An exploratory study}

            \vspace{0.10 cm}

        \mbox{\textbf{\textit{S. Varadarajan}}, S. Ladage}
    \end{twocolentry}

    \begin{onecolentry}
        \href{https://doi.org/10.18520/cs/v121/i3/354-359}{10.18520/cs/v121/i3/354-359}
    \end{onecolentry}
\end{samepage}



   \section{Conference Presentations}

\vspace{0.2cm}
\begin{itemize}
    \item \textbf{Exploring Higher Education (HE) Teachers’ Attempt to Multidisciplinary Curriculum Design}. Conference presentation at \textit{Episteme-10}, HBCSE, Mumbai, January 3–5, 2025.
    
    \item \textbf{Scaffolding Problem-Based Learning in Undergraduate Laboratories}. Oral presentation at \textit{X-Discipline Conference}, University of Nebraska–Lincoln, USA, March 1–3, 2021.
    
    \item \textbf{Practicing Science Skills through Virtual Chemistry Laboratory}. Poster presentation at \textit{Methods in Chemistry Education Research (MICER) 2021}, Keele University, UK, May 28, 2021. \\
    \textit{Co-authors: Anupa Kumbhar and Savita Ladage}
    
    \item \textbf{Problem-Based Learning: Life-cycle thinking approach to Indigo-dye lab}. Presented at \textit{IUPAC | CCCE 2021}, Canadian Chemistry Conference and Exhibition, August 13–20, 2021. \\
    \textit{Co-author: Savita Ladage}
    
    \item \textbf{Science Skills through Chemical Laboratories}. Presented during a \textit{ChemCollective Workshop on Virtual Laboratory}, organized by IIT Bombay for ~250 CBSE school teachers, June 17, 2021.
    
    \item \textbf{Chemistry Laboratory Education in the Digital Era}. Presented at \textit{Conventional and Digital Methods in Chemistry Education (CDMCE)}, organized by NIT Warangal, July 29–31, 2021.
    
    \item \textbf{Exploring the Role of Scaffolds in Mediating Reflective Thinking: Problem-Based Learning in an Undergraduate Chemistry Laboratory}. Presented at \textit{ECRICE-2020}, Weizmann Institute of Science, Israel.
    
    \item \textbf{Assessing Small-group Learning in a PBL Undergraduate Chemistry Laboratory}. Presented at \textit{MICER 2020}, University of Edinburgh, UK.
\end{itemize}
\section{Achievements}

\begin{onecolentry}
    \begin{highlights}
        \item Obtained Proficiency Prize in Inorganic Chemistry and Physics at the undergraduate level
        \item Secured University First Rank in the M.Sc. (Chemistry) program
        \item Awarded GATE-93 scholarship for research
        \item Founded the Funtasktic Science Club for primary and middle school students
        \item Received the Best Thesis Award from TIFR, Mumbai for Ph.D. research
    \end{highlights}
\end{onecolentry}



\section{References}

\begin{onecolentry}
    \begin{highlights}
        \vspace{0.2cm}
        \item \textbf{Prof. Diane M. Bunce} \\
        Professor Emerita, Department of Chemistry \\
        The Catholic University of America, Washington, DC \\
        \textit{Email:} \href{mailto:Bunce@cua.edu}{Bunce@cua.edu} \\
        \textit{Phone:} +1 410 279 3724

        \vspace{0.4cm}
        \item \textbf{Prof. Savita Ladage} \\
        Professor and Dean \\
        Homi Bhabha Centre for Science Education, TIFR, Mumbai \\
        \textit{Email:} \href{mailto:savital@hbcse.tifr.res.in}{savital@hbcse.tifr.res.in}, \href{mailto:savitaladage@gmail.com}{savitaladage@gmail.com} \\
        \textit{Phone:} +91 98202 50770
        \vspace{0.4cm}

        \item \textbf{Kalyani Gokhale} \\
        General Manager (Academics) \\
        Maharashtra State Faculty Development Academy, Pune, Maharashtra \\
        \textit{Email:} \href{mailto:kalyaniugokhale@gmail.com}{kalyaniugokhale@gmail.com} \\
        \textit{Phone:} +91 98900 25513

    \end{highlights}
\end{onecolentry}

    

\end{document}